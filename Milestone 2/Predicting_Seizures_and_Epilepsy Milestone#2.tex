%Jennifer Pan, August 2011

\documentclass[10pt,letter]{article}
	% basic article document class
	% use percent signs to make comments to yourself -- they will not show up.
\usepackage[letterpaper, portrait, margin=1in]{geometry}
\usepackage{amsmath}
\usepackage{amssymb}
	% packages that allow mathematical formatting

\usepackage{graphicx}
	% package that allows you to include graphics

\usepackage{setspace}
	% package that allows you to change spacing

\onehalfspacing
	% text become 1.5 spaced

\usepackage{fullpage}
	% package that specifies normal margins
	

\begin{document}
	% line of code telling latex that your document is beginning


\title{AC 209a: Milestone 2 Report}

\author{Xu Si,  Hung-Yi Wu, and James Zatsiorsky}

	% Note: when you omit this command, the current dateis automatically included
 
\maketitle 
	% tells latex to follow your header (e.g., title, author) commands.


\section{The Association Between Childhood Seizures and Later Childhood Emotional and Behavioral Problems}

\paragraph{} This paper studies the association of childhood seizures, divided into epilepsy, single unprovoked seizures and febrile
convulsions, with later childhood emotional/behavioral problems at age 7/11/16 years. Multivariate logistic regression and modified Poisson regression model are used to:
\begin{enumerate}
	\item evaluate the association of social disadvantages and FRIs (fetal risk indicators) with childhood seizures by age 7 years
	\item evaluate the association of childhood seizures by age 7 years with later childhood emotional/behavioral problems
	\item evaluate the association of childhood seizures by age 7 years with emotional/behavioral disorders in later childhood after accounting for social disadvantage and FRIs
\end{enumerate}

\paragraph{} The study shows that the association of epilepsy with emotional/behavioral problems, at 7/11/16 years, remains significant
even after accounting for indicators of social disadvantage and fetal life. Thus, early childhood epilepsy is worth
consideration in dealing with mental health problems. In addition, there is also a strong relationship between single
unprovoked seizures and emotional/behavioral problems at age 16 years.


\section{National Child Development Study}

\paragraph{} This paper discusses an extremely influential series of datasets (the 1958 birth cohort) that has led to changes in policy and practices on issues of human development, social inequalities, and health inequalities. The data has been used in over 900 publications in health and social science journals, and has led to key findings such as the link between smoking in pregnancy and health problems for the child.

\paragraph{} The datasets track a group of over 17,000 people from birth through the present day. Data was collected at various steps along the cohort's lifetime. Specifically, as of the writing of the report, data was collected at birth and subsequently at ages 7, 11, 16, 20, 23, 33, 42, 45. This extensive data coverage over eight different ages makes this dataset extremely valuable. At each age, we have medical reports for each person in the cohort, which will be useful in our analysis. Additionally, we have sociodemographic information collected at each age for the cohort.

\paragraph{} Importantly, the medical reports include survey information on \textit{Epileptic fits}. This information is included at ages 7, 11, 16, 23, 33, and 42. This data is critical to determining whether early seizures are any indication of seizures later in life. This journal article discusses some key findings and publications that came of this cohort's data, but findings on epileptic fits were not discussed here.

\paragraph{} There are a few key facts about the datasets that we need to keep in mind as we embark on our analysis. One is that the cohort does not demonstrate the diversity in ethnicity that we see in today's population. Additionally, there is a substantial sample loss over time. Therefore, even if we have interesting data for a person at age 11, it does not mean that we will have corresponding data for that person at ages 16, 23, etc.

\paragraph{} Fortunately, the data discussed in this article is available for non-commercial use free of charge. At age 45, a biomedical survey of the cohort was conducted, which includes genetic data. The report did not indicate whether or not this genetic data was publicly available, but it would be interesting to look into whether we can get access to this information. Then, we can see if there is a genetic link between having a first seizure and later developing epilepsy.

\section{Epilepsy Across the Spectrum}

\paragraph{} Epilepsy, broadly defined as the occurrence of two or more unprovoked seizures separated by at least 24 hours, are much more complex than a single disorder. The cause, severity, area of brain involved, locations and functions of body  affected are different from patient to patient. Up to 2010, more than 25 epilepsy syndromes and other epilepsy disorders have been delineated. As a result, it is better understood as a spectrum of disorders.

\paragraph{} Epilepsy is typically diagnosed by the patient or family members as incidence of seizures, or by the patient?s medical history (in ICD-CM codes). If medical history is used for analysis, it is important to be aware that some patients with epilepsy may choose not to seek medical care or may not have access to healthcare due to socioeconomic or health system barriers.

\paragraph{} Aside from the patient per se, epilepsy imposes direct and indirect cost to the socioeconomic circle around the patient, as well as the whole society. Direct cost includes medication and healthcare, while indirect cost includes loss of workforce. It is important to clarify whether the goal of analysis is to have higher epilepsy prediction accuracy or to minimize socioeconomic cost.

\paragraph{} Several disorders are found to related to developing epilepsy, including traumatic brain injury, brain tumor, stroke, central nervous system infection, autism spectrum disorders, Alzheimer?s disease, Rett syndrome, and tuberous sclerosis complex. Some disorders are found to increase the risk for developing epilepsy, neurocysticercosis, autism spectrum disorders, migraine, hypertension, behavioral problems like ADHD and depression. Some relations are bidirectional. For example, history of antidepressants treatment are shown to decrease incidence of seizures. Aside from disorders, preterm birth and genes like SCN1A also confer strong risks for developing epilepsy.

\paragraph{} The best dataset would be medical history, along with a response variable of where they were diagnosed to have epilepsy or not.


\end{document}
	% line of code telling latex that your document is ending. If you leave this out, you'll get an error
